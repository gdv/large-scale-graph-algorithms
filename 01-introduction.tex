\documentclass[12pt,aspectratio=169]{beamer}
\usepackage{pxfonts}

\usepackage{fancyvrb}
\fvset{%frame=single,
commandchars=\\\{\},
%framesep=1mm,
fontfamily=helvetica,
fontsize=\normalsize
}
%\usepackage[bitstream-charter]{mathdesign}
\usepackage{listings}
\lstset{commentstyle=\color{orange}, keywordstyle=\color{yellow} }

\usepackage{newpxtext}
\usepackage{eulerpx}
\usepackage{fontawesome5}

\usetheme{Boadilla}
\useoutertheme{split}
\usecolortheme{albatross}


\setbeamertemplate{blocks}[rounded][shadow=false]
\setbeamertemplate{navigation symbols}{}

\setbeamercolor*{structure}{fg=green!75!black,bg=blue!70!white}
\setbeamercolor*{normal text}{fg=green!65!black,bg=blue!80!black}
\setbeamercolor{palette primary}{use={structure,normal text},fg=green,bg=structure.bg!75!black}
\setbeamercolor{palette secondary}{use={structure,normal text},fg=structure.fg,bg=structure.bg!60!black}
\setbeamercolor{palette tertiary}{use={structure,normal text},fg=structure.fg,bg=structure.bg!45!black}
\setbeamercolor{palette quaternary}{use={structure,normal text},fg=green,bg=structure.bg!75!black}
\setbeamercolor*{example text}{fg=green!65!black}
\setbeamercolor*{block body}{bg=structure.bg!90!black}
\setbeamercolor*{block body alerted}{bg=structure.bg!90!black}
\setbeamercolor*{block body example}{bg=structure.bg!90!black}
\setbeamercolor*{block title}{parent=structure,bg=structure.bg!75!black}
\setbeamercolor*{block title alerted}{use={structure,alerted text},fg=alerted text.fg!75!structure.fg,bg=structure.bg!75!black}
\setbeamertemplate{navigation symbols}{}
\setbeamertemplate{items}[square]
\setbeamercolor{item projected}{fg=white}
\setbeamercolor*{normal text}{fg=white!90!blue,bg=blue!70!black}
\setbeamercolor*{separation line}{}
\setbeamercolor*{fine separation line}{}
\setbeamercolor{alerted text}{fg=green}

\usepackage[italian]{babel}
\usepackage[utf8]{inputenc}
\usepackage{pgf}
\usepackage{verbatim}
\usepackage{inconsolata}
\usepackage{listings}
\lstset{language=C, frame=single,
  basicstyle=\ttfamily,
  numbers=left, numberstyle=\tiny\color{gray},
  numbersep=5pt, fancyvrb=true
}
\usepackage[noend]{algorithm2e}

\usefonttheme{professionalfonts} % using non standard fonts for beamer
\usefonttheme{serif} % default family is serif
% \usepackage{fontspec}
% \setmainfont{Palatino}

\usepackage{pgf}
\usepackage{tikz}
\usepackage{graphicx}
\usetikzlibrary{%
  automata,
  arrows,
  arrows.meta,
  positioning,
  calc,
  backgrounds,
  chains,
  matrix,
  patterns,
  automata,
  fit,
  graphs,
  decorations,
  decorations.pathmorphing,
  decorations.pathreplacing,
  decorations.markings,
  shapes,
  shapes.geometric
}


\usepackage{url}
\usepackage{xmpmulti}
% \usepackage{euler}
%\usepackage[T1]{fontenc}
\pdfpagebox5
% \immediate\write18{sh ./vc}
% \input{vc}
\usepackage{transparent}
\graphicspath{{img/}}

\newcommand{\g}{\ensuremath{G=\langle V,E\rangle}\xspace}

\author{Gianluca Della Vedova}
\title{Large-Scale Graph Algorithms}
\institute{Univ. Milano--Bicocca\\
  \texttt{https://gianluca.dellavedova.org}}
\date{\today}
%\pgfdeclareimage[height=1cm]{university-logo}{logounimib}
%\logo{\pgfuseimage{university-logo}}


\begin{document}

\begin{frame}\frametitle{Gianluca Della Vedova}
\begin{itemize}
\item
Large-Scale Graph Algorithms
\item
Ufficio U14-2041
\item
    \url{https://www.unimib.it/gianluca-della-vedova}
\item
\url{gianluca.dellavedova@unimib.it}
\item\url{https://github.com/gdv/large-scale-graph-algorithms}
\item Everything at \url{https://elearning.unimib.it/}
\end{itemize}
\end{frame}

\begin{frame}[fragile]
\frametitle{Example}
\begin{columns}
  \begin{column}{0.45\textwidth}
    \includegraphics<1>[height=0.55\textheight]{img/Petersen_graph_3-coloring}
  \end{column}
  \begin{column}{0.45\textwidth}
      \begin{block}{Notation}
        \begin{itemize}
          \item
\alert{number of vertices}: $n$
          \item
\alert{number of edges/arcs}: $m$
        \end{itemize}
      \end{block}
  \end{column}
\end{columns}
\end{frame}

\begin{frame}[fragile]
\frametitle{Better representation}
    \includegraphics<1>{img/Caterpillar_tree}
      \begin{block}{Almost a path}
        \begin{itemize}
          \item
              Compact representation
        \end{itemize}
      \end{block}
\end{frame}

\begin{frame}[fragile]
\frametitle{Breadth-first visit}
\scalebox{.8}{
\begin{algorithm}[H]
\KwData{graph $G$, vertex $root$}
$Q\gets $ a queue\;
label root as explored\;
$Q$.enqueue(root)\;
\While{$Q\neq \emptyset$}{
  $v \gets Q$.dequeue()\;
    \ForEach{edge $(v,w)$}{
      \If{ $w$ is not labeled as explored}{
        label $w$ as explored\;
        $Q$.enqueue($w$)}
  }
}
\end{algorithm}}
\end{frame}


\begin{frame}[fragile]
\frametitle{Depth-first visit}
\scalebox{.8}{
\begin{algorithm}[H]
\KwData{graph $G$, vertex $root$}
$S\gets $ a stack\;
$S$.push(root)\;
\While{$S\neq \emptyset$}{
  $v \gets S$.pop()\;
  \If{ $v$ is not labeled as explored}{
    label $v$ as explored\;
    \ForEach{edge $(v,w)$}{
      $S$.push($w$)}
  }
}
\end{algorithm}}
\end{frame}


\begin{frame}[fragile]
\frametitle{Dijkstra's algorithm}
\scalebox{.8}{
\begin{algorithm}[H]
\KwData{graph $G$, vertex $source$}
$Q\gets $ a queue\;
\ForEach{vertex $v$}{
  dist[$v$] $\gets \infty$\;
  $Q$.enqueue(v)
  }
  dist[source] $\gets 0$\;
  \While{$Q \neq \emptyset$}{
    $u\gets $ vertex in $Q$ minimizing dist[$u$]\;
    $Q$.deque(u)\;
    \ForEach{ neighbor $v$ of $u$ still in $Q$}{
      $alt \gets dist[u] + Graph.Edges(u, v)$\;
      \If{ alt < dist[v]}{
        dist[v] $\gets$ alt\;
        prev[v] $\gets$ $u$\;}
    }}
  \Return dist[], prev[]\;
\end{algorithm}}
\end{frame}


\begin{frame}[fragile]
\frametitle{Dijkstra's algorithm --- priority queue}
\scalebox{.8}{
\begin{algorithm}[H]
\KwData{graph $G$, vertex $source$}
$Q\gets $ a priority queue\;
\ForEach{vertex $v$}{
  dist[$v$] $\gets \infty$\;
  $Q$.add\_with\_priority(v, dist[v])
  }
  dist[source] $\gets 0$\;
  \While{$Q \neq \emptyset$}{
    $u\gets Q$.extract\_min\;
    \ForEach{ neighbor $v$ of $u$ still in $Q$}{
      $alt \gets dist[u] + Graph.Edges(u, v)$\;
      \If{ alt < dist[v]}{
        dist[v] $\gets$ alt\;
        prev[v] $\gets$ $u$\;
        $Q$.decrease\_priority(v, alt)
        }
    }}
  \Return dist[], prev[]\;
\end{algorithm}}
\end{frame}


\begin{frame}[fragile]
\frametitle{Find articulation points}
\scalebox{.8}{
\begin{algorithm}[H]
\KwData{connected graph $G$, vertex $root$}
$S\gets $ a stack\;
$S$.push((root, nil))\;
$d\gets -1$\;
\While{$S\neq \emptyset$}{
  $d\gets d+1$\;
  $(v, p) \gets S$.peek()\;
  \eIf{$v$ is not explored}{
    label $v$ as explored; parent$(v)\gets p$; $depth[v]\gets d$\;
    lowpoint$[v] = $depth$[v] $\;
    \ForEach{edge $(v,w)$}{
      $S$.push($(w,v)$)\;
    }
  }{
lowpoint$[v] = \min \left\{ \min_{w\in N(v), w\neq p} \{ depth[w] \}, \min_{w\in N(v), parent(w)=v} \{ lowpoint[w] \}  \right\}$
  }
    $d\gets d-1$\;
    $v \gets S$.pop()\;
  }
\end{algorithm}}
\end{frame}



\begin{frame}[fragile]
\frametitle{Find bridges}
\scalebox{.8}{
\begin{algorithm}[H]
\KwData{connected graph $G$, vertex $root$}
$S\gets $ a stack\;
$S$.push((root, nil))\;
$x\gets -1$\;
\While{$S\neq \emptyset$}{
  $(v, p) \gets S$.peek()\;
  \eIf{$v$ is not explored}{
    \(x\gets x+1\)\;
    label $v$ as explored; P$(v)\gets p$; \(num[v]\gets x\)\;
    \ForEach{edge $(v,w)$}{
      $S$.push($(w,v)$)\;
    }
  }{
   $nd[v] = 1 + \sum_{w\in N(v), P(w)=v} nd[w]$\;
  $l[v] = \min  \left\{ num[v], \min_{w\in N(v), w\neq p, P(w)\neq v} \{ num[w] \}, \min_{w\in children(v)} \{ l[w] \}  \right\}$\;
  $h[v] = \max  \left\{ num[v], \max_{w\in N(v), w\neq p, P(w)\neq v} \{ num[w] \}, \max_{w\in children(v)} \{ h[w] \}  \right\}$\;
  }
    $v \gets S$.pop()\;
  }
\end{algorithm}}
\end{frame}



\begin{frame}[fragile]
\frametitle{Max flow}

\tikzstyle{vertex}=[circle,draw,minimum size=20pt,inner sep=0pt]
\tikzstyle{selected vertex} = [vertex, fill=black!80]
\tikzstyle{sourcesink} = [vertex, accepting]
\tikzstyle{edge} = [draw,thick,->]
\tikzstyle{weight} = [font=\small]
\tikzstyle{selected edge} = [draw,line width=5pt,-,red!50]
\tikzstyle{ignored edge} = [draw,line width=5pt,-,black!20]

\begin{figure}
\begin{tikzpicture}[scale=1.8, auto,swap]
    % Draw a 7,11 network
    % First we draw the vertices
    \foreach \pos/\name in {{(0,2)/a}, {(2,2)/b}, {(5,2)/c},
      {(0,0)/d}, {(2.5,0.5)/e}, {(2,-1)/f}, {(4,-1)/g}}
       \node[vertex] (\name) at \pos {$\name$};
    \foreach \pos/\name in {{(-2,1)/s}, {(6,0)/t}}
       \node[sourcesink] (\name) at \pos {$\name$};
       % Connect vertices with edges and draw weights
       \foreach \source/ \dest /\weight in {a/b/(7), b/c/(8),
         d/a/(5),d/b/(9),
                                         e/b/(7), e/c/(5),
                                         d/f/(6), f/e/(8),
                                         g/e/(9), f/g/(11),
                                         s/a/(9), s/d/(8),
                                       c/t/(10), e/t/(6), g/t/(6) }
        \path[edge] (\source) -- node[weight] {$\weight$} (\dest);
    % Start animating the vertex and edge selection.
%    \foreach \vertex / \fr in {d/1,a/2,f/3,b/4,e/5,c/6,g/7}
%        \path<\fr-> node[selected vertex] at (\vertex) {$\vertex$};
    % For convenience we use a background layer to highlight edges
    % This way we don't have to worry about the highlighting covering
    % weight labels.
%    \begin{pgfonlayer}{background}
%        \pause
%        \foreach \source / \dest in {d/a,d/f,a/b,b/e,e/c,e/g}
%            \path<+->[selected edge] (\source.center) -- (\dest.center);
%        \foreach \source / \dest / \fr in {d/b/4,d/e/5,e/f/5,b/c/6,f/g/7}
%            \path<\fr->[ignored edge] (\source.center) -- (\dest.center);
%    \end{pgfonlayer}
\end{tikzpicture}
\end{figure}

\end{frame}


\begin{frame}[fragile]
\frametitle{Max flow -- Min cut theorem}


Let \(f\) be a flow of a graph \(G=(V,E)\).
Then the following three conditions are equivalent:

\begin{enumerate}
\item
      \(f\) is a maximum flow
\item
      the residual graph has no augmenting path
\item
      there is a cut \((S,T)\) of \(G\) such that \(c(S,T) = |f|\)
\end{enumerate}
\end{frame}

\begin{frame}[fragile]
\frametitle{Figures}

        \begin{itemize}
          \item
David Eppstein, Public Domain, \url{https://commons.wikimedia.org/w/index.php?curid=10261635}
        \end{itemize}
\end{frame}

\begin{frame}[containsverbatim]\frametitle{License}
  \small
This work is licensed under a Creative Commons Attribution-ShareAlike
4.0 International License
(\url{https://creativecommons.org/licenses/by-sa/4.0/}).

\faCreativeCommons\ \faCreativeCommonsBy\ \faCreativeCommonsSa
\end{frame}
\end{document}
