\documentclass[12pt,aspectratio=169]{beamer}
\usepackage{pxfonts}

\usepackage{fancyvrb}
\fvset{%frame=single,
commandchars=\\\{\},
%framesep=1mm,
fontfamily=helvetica,
fontsize=\normalsize
}
%\usepackage[bitstream-charter]{mathdesign}
\usepackage{listings}
\lstset{commentstyle=\color{orange}, keywordstyle=\color{yellow} }

\usepackage{newpxtext}
\usepackage{eulerpx}
\usepackage{fontawesome5}

\usetheme{Boadilla}
\useoutertheme{split}
\usecolortheme{albatross}


\setbeamertemplate{blocks}[rounded][shadow=false]
\setbeamertemplate{navigation symbols}{}

\setbeamercolor*{structure}{fg=green!75!black,bg=blue!70!white}
\setbeamercolor*{normal text}{fg=green!65!black,bg=blue!80!black}
\setbeamercolor{palette primary}{use={structure,normal text},fg=green,bg=structure.bg!75!black}
\setbeamercolor{palette secondary}{use={structure,normal text},fg=structure.fg,bg=structure.bg!60!black}
\setbeamercolor{palette tertiary}{use={structure,normal text},fg=structure.fg,bg=structure.bg!45!black}
\setbeamercolor{palette quaternary}{use={structure,normal text},fg=green,bg=structure.bg!75!black}
\setbeamercolor*{example text}{fg=green!65!black}
\setbeamercolor*{block body}{bg=structure.bg!90!black}
\setbeamercolor*{block body alerted}{bg=structure.bg!90!black}
\setbeamercolor*{block body example}{bg=structure.bg!90!black}
\setbeamercolor*{block title}{parent=structure,bg=structure.bg!75!black}
\setbeamercolor*{block title alerted}{use={structure,alerted text},fg=alerted text.fg!75!structure.fg,bg=structure.bg!75!black}
\setbeamertemplate{navigation symbols}{}
\setbeamertemplate{items}[square]
\setbeamercolor{item projected}{fg=white}
\setbeamercolor*{normal text}{fg=white!90!blue,bg=blue!70!black}
\setbeamercolor*{separation line}{}
\setbeamercolor*{fine separation line}{}
\setbeamercolor{alerted text}{fg=green}

\usepackage[italian]{babel}
\usepackage[utf8]{inputenc}
\usepackage{pgf}
\usepackage{verbatim}
\usepackage{inconsolata}
\usepackage{listings}
\lstset{language=C, frame=single,
  basicstyle=\ttfamily,
  numbers=left, numberstyle=\tiny\color{gray},
  numbersep=5pt, fancyvrb=true
}
\usepackage[noend]{algorithm2e}

\usefonttheme{professionalfonts} % using non standard fonts for beamer
\usefonttheme{serif} % default family is serif
% \usepackage{fontspec}
% \setmainfont{Palatino}

\usepackage{pgf}
\usepackage{tikz}
\usepackage{graphicx}
\usetikzlibrary{%
  automata,
  arrows,
  arrows.meta,
  positioning,
  calc,
  backgrounds,
  chains,
  matrix,
  patterns,
  automata,
  fit,
  graphs,
  decorations,
  decorations.pathmorphing,
  decorations.pathreplacing,
  decorations.markings,
  shapes,
  shapes.geometric
}


\usepackage{url}
\usepackage{xmpmulti}
% \usepackage{euler}
%\usepackage[T1]{fontenc}
\pdfpagebox5
% \immediate\write18{sh ./vc}
% \input{vc}
\usepackage{transparent}
\graphicspath{{img/}}

\newcommand{\g}{\ensuremath{G=\langle V,E\rangle}\xspace}

\author{Gianluca Della Vedova}
\title{Large-Scale Graph Algorithms}
\institute{Univ. Milano--Bicocca\\
  \texttt{https://gianluca.dellavedova.org}}
\date{\today}
%\pgfdeclareimage[height=1cm]{university-logo}{logounimib}
%\logo{\pgfuseimage{university-logo}}


\begin{document}



\begin{frame}[fragile]
	\frametitle{Max flow}

	\tikzstyle{vertex}=[circle,draw,minimum size=20pt,inner sep=0pt]
	\tikzstyle{selected vertex} = [vertex, fill=black!80]
	\tikzstyle{sourcesink} = [vertex, accepting, color=green, fill=blue!90]
	\tikzstyle{edge} = [draw,thick,->, >=latex]
	\tikzstyle{weight} = [font=\small]
	\tikzstyle{selected edge} = [draw,line width=5pt,-,red!50]
	\tikzstyle{ignored edge} = [draw,line width=5pt,-,black!20]

	\begin{figure}
		\begin{tikzpicture}[scale=1.8, auto,swap]
			% Draw a 7,11 network
			% First we draw the vertices
			\foreach \pos/\name in {{(0,2)/a}, {(2,2)/b}, {(5,2)/c},
					{(0,0)/d}, {(2.5,0.5)/e}, {(2,-1)/f}, {(4,-1)/g}}
			\node[vertex] (\name) at \pos {$\name$};
			\foreach \pos/\name in {{(-2,1)/s}, {(6,0)/t}}
			\node[sourcesink] (\name) at \pos {$\name$};
			% Connect vertices with edges and draw weights
			\foreach \source/ \dest /\weight in {a/b/(7), b/c/(8),
					d/a/(5),d/b/(9),
					e/b/(7), e/c/(5),
					d/f/(6), f/e/(8),
					g/e/(9), f/g/(11),
					s/a/(9), s/d/(8),
					c/t/(10), e/t/(6), g/t/(6) }
			\path[edge] (\source) -- node[weight] {$\weight$} (\dest);
			% Start animating the vertex and edge selection.
			%    \foreach \vertex / \fr in {d/1,a/2,f/3,b/4,e/5,c/6,g/7}
			%        \path<\fr-> node[selected vertex] at (\vertex) {$\vertex$};
			% For convenience we use a background layer to highlight edges
			% This way we don't have to worry about the highlighting covering
			% weight labels.
			%    \begin{pgfonlayer}{background}
			%        \pause
			%        \foreach \source / \dest in {d/a,d/f,a/b,b/e,e/c,e/g}
			%            \path<+->[selected edge] (\source.center) -- (\dest.center);
			%        \foreach \source / \dest / \fr in {d/b/4,d/e/5,e/f/5,b/c/6,f/g/7}
			%            \path<\fr->[ignored edge] (\source.center) -- (\dest.center);
			%    \end{pgfonlayer}
		\end{tikzpicture}
	\end{figure}

\end{frame}

\begin{frame}[fragile]
	\frametitle{Max flow}

	\begin{columns}
		\begin{column}{0.45\textwidth}
	\tikzstyle{vertex}=[circle,draw,minimum size=20pt,inner sep=0pt]
	\tikzstyle{selected vertex} = [vertex, fill=black!80]
	\tikzstyle{sourcesink} = [vertex, accepting, color=green, fill=blue!90]
	\tikzstyle{edge} = [draw,thick,->, >=latex]
	\tikzstyle{weight} = [font=\small]
	\tikzstyle{selected edge} = [draw,line width=5pt,-,red!50]
	\tikzstyle{ignored edge} = [draw,line width=5pt,-,black!20]

	\begin{figure}

    \begin{tikzpicture}[auto, node distance=2cm, yscale=.6,
      xscale=2.2, scale=0.7, every node/.style={scale=.7}]
      \tikzstyle{edge_style} =  [draw=black, line  width=2, thick,
      ->, >=latex]
      \tikzstyle{marked} = [draw=red, line width=5,  opacity=.3]
      \tikzstyle{potential} = [draw=blue, line width=5, opacity=.3]
      \node[sourcesink] (s) at (-2.5,0) {s};
      \node[vertex] (1) at (-2,2) {1};
      \node[vertex] (2) at (-1.5,0) {2};
      \node[vertex] (3) at (-2,-2) {3};
      \node[vertex] (4) at (0.8,2) {4};
      \node[vertex] (5) at (-0.6,0.5) {5};
      \node[sourcesink] (t) at (0.5,-2) {t};
      \node (s_) at (-3.2,0) {};
      \node (t_) at (0.5,-3.5) {};
      \node (4_) at (1.5,2) {};

      %% EDGES
      \foreach \source/\dest/\w/\pos in {
        s/1/2/-> , s/2/6/-> , s/3/5/left, 1/4/t/->,
        1/5/4/->, 2/5/3/->, 2/t/4/below, 3/t/3/below, 4/t/4/-> , 5/t/2/->
      }
      \draw[edge] (\source) edge node[\pos]{} (\dest);

      \draw[edge] (1) edge node{$f_{ij} (c_{ij})$} (4);
		\end{tikzpicture}
	\end{figure}
	\end{column}
	\begin{column}{0.55\textwidth}
  \begin{itemize}
    \item $c_{ij}$ : capacity of the arc $(i,j)$
    \item $f_{ij}$ : flow through the arc $(i,j)$
    \item $v_i$ : flow imbalance of the vertex $i$ ($<0$ incoming, $>0$ outgoing)
    \end{itemize}
	\end{column}
	\end{columns}
	\begin{align*}
    \max\quad F \qquad\qquad\qquad\qquad&\\
    \text{s.t } \sum_{j:(i,j)\in
    E}f_{ij}-\sum_{k:(i,k)\in E}f_{ki}&=
                                        \begin{cases}
                                          F& \text{if } i=s\\
                                          -F& \text{if } i=t\\
                                          0& i \text{ otherwise}
                                        \end{cases}
                                      && \forall i\in V\\
    0\le f_{ij}&\le c_{ij} &&\forall (i,j)\in E
  \end{align*}
	\end{frame}

\begin{frame}[fragile]{Example}

	\tikzstyle{vertex}=[circle,draw,minimum size=20pt,inner sep=0pt]
	\tikzstyle{selected vertex} = [vertex, fill=black!80]
	\tikzstyle{sourcesink} = [vertex, accepting, color=green, fill=blue!90]
	\tikzstyle{edge} = [draw,thick,->, >=latex]
	\tikzstyle{weight} = [font=\small]
	\tikzstyle{selected edge} = [draw,line width=5pt,-,red!50]
	\tikzstyle{ignored edge} = [draw,line width=5pt,-,black!20]

	\begin{figure}
		\begin{tikzpicture}[scale=1.8, auto,swap]
			% Draw a 7,11 network
			% First we draw the vertices
			\foreach \pos/\name in {{(0,2)/a}, {(2,2)/b}, {(5,2)/c},
					{(0,0)/d}, {(2.5,0.5)/e}, {(2,-1)/f}, {(4,-1)/g}}
			\node[vertex] (\name) at \pos {$\name$};
			\foreach \pos/\name in {{(-2,1)/s}, {(6,0)/t}}
			\node[sourcesink] (\name) at \pos {$\name$};
			% Connect vertices with edges and draw weights
			\foreach \source/\dest/\capacity/\position in {
			  s/a/(5)/above, s/d/(8)/->,
			  a/b/(3)/above, a/d/(3)/->,
			  b/c/(8)/above, b/d/(9)/->,
			  c/t/(7)/above right,  c/e/(5)/->,
			  d/f/(10)/->, d/e/(4)/->,
			  e/t/(7)/above, e/b/(7)/->,
			  f/g/(4)/->, f/e/(8)/->,
			  g/e/(9)/->,
			  t/g/(6)/->
			}
			\draw[edge] (\source) edge node[\position]{\capacity} (\dest);
			% Start animating the vertex and edge selection.
			%    \foreach \vertex / \fr in {d/1,a/2,f/3,b/4,e/5,c/6,g/7}
			%        \path<\fr-> node[selected vertex] at (\vertex) {$\vertex$};
			% For convenience we use a background layer to highlight edges
			% This way we don't have to worry about the highlighting covering
			% weight labels.
			%    \begin{pgfonlayer}{background}
			%        \pause
			%        \foreach \source / \dest in {d/a,d/f,a/b,b/e,e/c,e/g}
			%            \path<+->[selected edge] (\source.center) -- (\dest.center);
			%        \foreach \source / \dest / \fr in {d/b/4,d/e/5,e/f/5,b/c/6,f/g/7}
			%            \path<\fr->[ignored edge] (\source.center) -- (\dest.center);
			%    \end{pgfonlayer}
		\end{tikzpicture}
	\end{figure}

\end{frame}



\begin{frame}[fragile]
	\frametitle{Residual network}
	\tikzstyle{vertex}=[circle,draw,minimum size=20pt,inner sep=0pt]
	\tikzstyle{selected vertex} = [vertex, fill=black!80]
	\tikzstyle{sourcesink} = [vertex, accepting, color=green, fill=blue!90]
	\tikzstyle{edge_style} = [draw,thick,->, >=latex]
	\tikzstyle{weight} = [font=\small]
	\tikzstyle{selected edge} = [draw,line width=5pt,-,red!50]
	\tikzstyle{ignored edge} = [draw,line width=5pt,-,black!20]
    \tikzstyle{marked} = [draw=red, line width=5,  opacity=.3]
    \tikzstyle{unmarked} = [draw=gray, line width=5, opacity=.3]
    \tikzstyle{potential} = [draw=green, line width=5, opacity=.3]

    \begin{tikzpicture}[auto, node distance=2cm]

    \node[sourcesink] (s) at (0,-1) {s};
    \node[vertex] (1) at (2,2) {a};
    \node[vertex] (2) at (3,-2) {b};
    \node[sourcesink] (t) at (5,1) {t};

     \foreach \source/\dest/\w/\c/\pos in {
      s/1/10/10/-> , s/2/0/5/below left, 1/t/0/5/->, 2/t/10/10/below right, 1/2/10/15/->
    }
    \onslide<1-3>\draw[edge_style] (\source) edge node[\pos]{\c (\w)} (\dest);

	\onslide<2>\draw[marked] (s)--(1)--(2)--(t);

     \foreach \source/\dest/\w/\pos in {
      s/1/(0)/-> , s/2/(5)/below, 1/t/(5)/->, 2/t/(0)/below, 1/2/(5)/->
    }
    \onslide<4>\draw[edge_style] (\source) edge node[\pos]{\w} (\dest);

    \onslide<3-4>\draw[edge_style, green, thick] (2) edge[bend left=15] node{\color{green}(10)} (1);
    \onslide<3-4>\draw[edge_style, green, thick] (1) edge[bend right=90] node[left]{\color{green}(10)} (s);
    \onslide<3-4>\draw[edge_style, green, thick] (t) edge[bend left=90] node[right]{\color{green}(10)} (2);
  \end{tikzpicture}
  \end{frame}

\begin{frame}[fragile]{Cut}

	Let \g be a directed graph and let $S\subseteq V$.
	Then:
	\begin{itemize}
		  \item
$(S, \bar{S}) = E \cap (S \times \bar{S})$ is a forward cut,
		  \item
$(\bar{S}, S) = E \cap (\bar{S} \times S)$ is a backward cut,
		  \item
$E \cap \left( (S \times \bar{S}) \cup (\bar{S} \times S)\right)$ is a cut.
	\end{itemize}
	\end{frame}



\begin{frame}[fragile]{Flow --- Cut}

	\begin{block}{Lemma 1}
	Let \g be a directed graph and
let $(S, \bar{S})$ be a bipartition of $V$, with $s\in S$, $t\notin S$.
Let $f$ be an $(s,t)$-flow with total flow $F$.
Then
	\begin{equation*}
F = \sum_{e\in (S, \bar{S})} f(e) -  \sum_{e\in (\bar{S}, S)} f(e)
\end{equation*}
\end{block}

\begin{block}{Lemma 2}
	Let \g be a directed graph and
let $(S, \bar{S})$ be a bipartition of $V$, with $s\in S$, $t\notin S$.
Let $f$ be an $(s,t)$-flow with total flow $F$.
Then
	\begin{equation*}
F \le \sum_{e\in (S, \bar{S})} c(e)
\end{equation*}
\end{block}
\end{frame}

\begin{frame}[fragile]
	\frametitle{Max flow -- Min cut theorem}

	Let \(f\) be a flow of a graph \(G=(V,E)\).
	Then the following three conditions are equivalent:

	\begin{enumerate}
		\item
		      \(f\) is a maximum flow
		\item
		      the residual graph has no \alert{augmenting} path
		\item
		      there is a cut \((S,T)\) of \(G\) such that \(c(S,T) = |f|\)
	\end{enumerate}
	\end{frame}


	\begin{frame}[fragile]
	\frametitle{Ford--Fulkerson}
	\begin{columns}
		\begin{column}{0.45\textwidth}
	\begin{enumerate}
		\item
		  Find an augmenting path
	\item
		  Use it!
	\end{enumerate}
	\onslide<3>\begin{block}{Edmonds--Karp}
	BFS to find the augmenting path
	\end{block}
	\end{column}
	\begin{column}{0.45\textwidth}
		\tikzstyle{vertex}=[circle,draw,minimum size=20pt,inner sep=0pt]
	\tikzstyle{selected vertex} = [vertex, fill=black!80]
	\tikzstyle{sourcesink} = [vertex, accepting, color=green, fill=blue!90]
	\tikzstyle{edge_style} = [draw,thick,->, >=latex]
	\tikzstyle{weight} = [font=\small]
	\tikzstyle{selected edge} = [draw,line width=5pt,-,red!50]
	\tikzstyle{ignored edge} = [draw,line width=5pt,-,black!20]
    \tikzstyle{marked} = [draw=red, line width=5,  opacity=.3]
    \tikzstyle{unmarked} = [draw=gray, line width=5, opacity=.3]
    \tikzstyle{potential} = [draw=green, line width=5, opacity=.3]

    \onslide<2->\begin{tikzpicture}[auto, node distance=2cm]

    \node[sourcesink] (s) at (0,-1) {s};
    \node[vertex] (1) at (2,2) {a};
    \node[vertex] (2) at (3,-2) {b};
    \node[sourcesink] (t) at (5,1) {t};

     \foreach \source/\dest/\f/\c/\pos in {
	   s/1/10/99/->,
	   s/2/0/99/below left,
	   1/t/0/99/->,
	   2/t/10/99/below right,
	   1/2/10/1/->
    }
	\draw[edge_style] (\source) edge node[\pos]{(\c)} (\dest);
	\end{tikzpicture}
	\end{column}
	\end{columns}
	\end{frame}

\begin{frame}[fragile]
	\frametitle{Figures}

	\begin{itemize}
		\item
		      David Eppstein, Public Domain, \url{https://commons.wikimedia.org/w/index.php?curid=10261635}
	\end{itemize}
\end{frame}

\begin{frame}[containsverbatim]\frametitle{License}
  \small
This work is licensed under a Creative Commons Attribution-ShareAlike
4.0 International License
(\url{https://creativecommons.org/licenses/by-sa/4.0/}).

\faCreativeCommons\ \faCreativeCommonsBy\ \faCreativeCommonsSa
\end{frame}
\end{document}
