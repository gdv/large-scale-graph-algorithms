\documentclass[12pt,aspectratio=169]{beamer}
\usepackage{pxfonts}

\usepackage{fancyvrb}
\fvset{%frame=single,
commandchars=\\\{\},
%framesep=1mm,
fontfamily=helvetica,
fontsize=\normalsize
}
%\usepackage[bitstream-charter]{mathdesign}
\usepackage{listings}
\lstset{commentstyle=\color{orange}, keywordstyle=\color{yellow} }

\usepackage{newpxtext}
\usepackage{eulerpx}
\usepackage{fontawesome5}

\usetheme{Boadilla}
\useoutertheme{split}
\usecolortheme{albatross}


\setbeamertemplate{blocks}[rounded][shadow=false]
\setbeamertemplate{navigation symbols}{}

\setbeamercolor*{structure}{fg=green!75!black,bg=blue!70!white}
\setbeamercolor*{normal text}{fg=green!65!black,bg=blue!80!black}
\setbeamercolor{palette primary}{use={structure,normal text},fg=green,bg=structure.bg!75!black}
\setbeamercolor{palette secondary}{use={structure,normal text},fg=structure.fg,bg=structure.bg!60!black}
\setbeamercolor{palette tertiary}{use={structure,normal text},fg=structure.fg,bg=structure.bg!45!black}
\setbeamercolor{palette quaternary}{use={structure,normal text},fg=green,bg=structure.bg!75!black}
\setbeamercolor*{example text}{fg=green!65!black}
\setbeamercolor*{block body}{bg=structure.bg!90!black}
\setbeamercolor*{block body alerted}{bg=structure.bg!90!black}
\setbeamercolor*{block body example}{bg=structure.bg!90!black}
\setbeamercolor*{block title}{parent=structure,bg=structure.bg!75!black}
\setbeamercolor*{block title alerted}{use={structure,alerted text},fg=alerted text.fg!75!structure.fg,bg=structure.bg!75!black}
\setbeamertemplate{navigation symbols}{}
\setbeamertemplate{items}[square]
\setbeamercolor{item projected}{fg=white}
\setbeamercolor*{normal text}{fg=white!90!blue,bg=blue!70!black}
\setbeamercolor*{separation line}{}
\setbeamercolor*{fine separation line}{}
\setbeamercolor{alerted text}{fg=green}

\usepackage[italian]{babel}
\usepackage[utf8]{inputenc}
\usepackage{pgf}
\usepackage{verbatim}
\usepackage{inconsolata}
\usepackage{listings}
\lstset{language=C, frame=single,
  basicstyle=\ttfamily,
  numbers=left, numberstyle=\tiny\color{gray},
  numbersep=5pt, fancyvrb=true
}
\usepackage[noend]{algorithm2e}

\usefonttheme{professionalfonts} % using non standard fonts for beamer
\usefonttheme{serif} % default family is serif
% \usepackage{fontspec}
% \setmainfont{Palatino}

\usepackage{pgf}
\usepackage{tikz}
\usepackage{graphicx}
\usetikzlibrary{%
  automata,
  arrows,
  arrows.meta,
  positioning,
  calc,
  backgrounds,
  chains,
  matrix,
  patterns,
  automata,
  fit,
  graphs,
  decorations,
  decorations.pathmorphing,
  decorations.pathreplacing,
  decorations.markings,
  shapes,
  shapes.geometric
}


\usepackage{url}
\usepackage{xmpmulti}
% \usepackage{euler}
%\usepackage[T1]{fontenc}
\pdfpagebox5
% \immediate\write18{sh ./vc}
% \input{vc}
\usepackage{transparent}
\graphicspath{{img/}}

\newcommand{\g}{\ensuremath{G=\langle V,E\rangle}\xspace}

\author{Gianluca Della Vedova}
\title{Large-Scale Graph Algorithms}
\institute{Univ. Milano--Bicocca\\
  \texttt{https://gianluca.dellavedova.org}}
\date{\today}
%\pgfdeclareimage[height=1cm]{university-logo}{logounimib}
%\logo{\pgfuseimage{university-logo}}


\begin{document}

% \begin{frame}[fragile]
%     \frametitle{Graph Compression}
% \begin{columns}
%   \begin{column}{0.45\textwidth}
%     \includegraphics<1>[height=0.55\textheight]{img/Petersen_graph_3-coloring}
%   \end{column}
%   \begin{column}{0.45\textwidth}
%       \begin{block}{Notation}
%         \begin{itemize}
%           \item
% \alert{number of vertices}: $n$
%           \item
% \alert{number of edges/arcs}: $m$
%         \end{itemize}
%       \end{block}
%   \end{column}
% \end{columns}
% \end{frame}
%
\begin{frame}[fragile]
\frametitle{Lossless Graph Compression}

\tikzstyle{vertex}=[circle,draw,minimum size=20pt,inner sep=0pt]
\tikzstyle{edge} = [draw,thick,-]
\begin{figure}
\begin{tikzpicture}[scale=1.8, auto,swap]
    % Draw a 7,11 network
    % First we draw the vertices
    \foreach \pos/\name/\label in {{(1,2)/a/a}, {(4,2)/b/b}, {(0,0)/d/g},
      {(1,-1)/c/f}, {(5,0)/e/d}, {(4,-1)/f/e}, {(0,1)/g/h},
      {(5,1)/h/c}}
        \node[vertex] (\name) at \pos {$\label$};
    % Connect vertices with edges and draw weights
    \foreach \source/ \dest in {b/a, d/b,c/a,c/b,
                                c/d,     e/d,e/c,
                                        d/f, f/e, f/g,
                                         f/h, g/e,g/f, g/h}
        \path[edge] (\source) --  (\dest);
\end{tikzpicture}
\end{figure}
\end{frame}

\begin{frame}[fragile]
\frametitle{Lossless Graph Compression}
\begin{block}{Main ideas}
\begin{itemize}
\item
      Exploit repetitions
\item
      Exploit distribution of values
\end{itemize}
\end{block}
\end{frame}



\begin{frame}[fragile]
\frametitle{Huffman coding}
\begin{algorithm}[H]
\KwData{Set \(O\) of objects, each object \(o_{i}\) has probability \(p_{i}\)}
\eIf{$|O| > 2$}{
  Pick two objects \(o_{i}, o_{j}\) with smallest probability\;
  \(x\gets \) new object with probability \(p_{i} + p_{j}\)\;
  \(h \gets \) Huffman\((O\setminus \{o_{i}, o_{j}\} \cup \{x\})\)\;
  \(h(o_{i}) \gets h(x) 0\)\;
  \(h(o_{j}) \gets h(x) 1\)\;
  Remove \(h(x)\)\;
}{
  \(h(o_{1}) \gets 0\)\;
  \(h(o_{2}) \gets 1\)\;
}
  \Return \(h\)\;
\end{algorithm}
\end{frame}


\begin{frame}[fragile]
\frametitle{Elias \(\gamma\) code}
\begin{block}{binary code for \(x\ge 1\)}
\begin{enumerate}
\item
      \(N= \lfloor \log_{2} x\rfloor\)
\item
      \(N\) zeroes \(\cdot\) 1 one  \(\cdot\) binary representation of \(x\),
      omitting the leading bit
\item
      uses \(2\lfloor \log_{2} x\rfloor + 1\) bits
\end{enumerate}
\end{block}
\end{frame}


\begin{frame}[fragile]
\frametitle{Elias \(\delta\) code}
\begin{block}{binary code for \(x\ge 1\)}
\begin{enumerate}
\item
      \(N= \lfloor \log_{2} x\rfloor\)
\item
      \(\gamma(N+1)\) \(\cdot\) binary representation of \(x\), omitting the
      leading bit
\item
      uses
      \(\lfloor \log_{2} x\rfloor + 2 \lfloor \log_{2} \left(\lfloor \log_{2} x\rfloor + 1 \right)\rfloor
      +1\) bits
\end{enumerate}
\end{block}
\end{frame}


\begin{frame}[fragile]
\frametitle{Variable-length nibble code}
\begin{block}{binary code for \(x\ge 1\)}
\begin{enumerate}
\item
      \(p \gets\) the binary representation of \(n\), left-padded with zeroes,
      so that its length is a multiple of 3
\item
      Split \(p\) into 3-bit blocks
\item
      prepend each block with a zero, replace the leading 0 of the last block with
      a one.
\end{enumerate}
\end{block}
\end{frame}


\begin{frame}[fragile]
\frametitle{Minimal binary code}
\begin{block}{binary code for \(0\le x\le z-1\)}
\begin{enumerate}
\item
\(s = \lceil \log_{2}x \rceil\)
\item
      \(p \gets s^{s}-z\)
\item
      If \(x < p\) then output the \(x\)-th binary word of length \(s-1\)
\item
      If \(x \ge p\) then output the \((x-z+2^{s})\)-th binary word of length \(s\)
\end{enumerate}
\end{block}
\end{frame}

\begin{frame}[fragile]
\frametitle{\(\zeta_{k}\) code}
\begin{block}{binary code for \(2^{hk}\le x\le 2^{(h+1)k}-1\)}
\begin{enumerate}
\item
      \(k\): shrinking factor
\item
      h+1 in unary \(\cdot\) minbincode of \(x - 2^{hk}\), with
\(z = 2^{(h+1)k} - 2^{hk} -1\)
\end{enumerate}
\end{block}
\end{frame}

\begin{frame}[fragile]
\frametitle{Move-to-front transform}

\begin{enumerate}
\item
      Maintain the list \(L\) of recently used objects
\item
      Encode an object as its index in \(L\)
\item
      Move the object to the head of \(L\)
\end{enumerate}
\end{frame}



\begin{frame}[fragile]
\frametitle{Run-length encoding}

AAAAAABBBB \(\rightarrow\) (A,6)(B,4)

\only<2>{
\begin{block}{Binary alphabet}
\begin{itemize}
\item
      0-runs and 1-runs alternate
\item
      start with 1-run (0-length if 0-run)
\item
      first, number of runs
\item
      then run-lengths (lengths decremented by 1)
\end{itemize}
\end{block}}
\end{frame}

\begin{frame}[fragile]
\frametitle{Lossless Graph Compression}

\tikzstyle{vertex}=[circle,draw,minimum size=20pt,inner sep=0pt]
\tikzstyle{edge} = [draw,thick,-]
\begin{figure}
\begin{tikzpicture}[scale=1.8, auto,swap]
    % Draw a 7,11 network
    % First we draw the vertices
    \foreach \pos/\name/\label in {{(1,2)/a/a}, {(4,2)/b/b}, {(0,0)/d/g},
      {(1,-1)/c/f}, {(5,0)/e/d}, {(4,-1)/f/e}, {(0,1)/g/h},
      {(5,1)/h/c}}
        \node[vertex] (\name) at \pos {$\label$};
    % Connect vertices with edges and draw weights
    \foreach \source/ \dest in {b/a, d/b,c/a,c/b,
                                c/d,     e/d,e/c,
                                        d/f, f/e, f/g,
                                         f/h, g/e,g/f, g/h}
        \path[edge] (\source) --  (\dest);
\end{tikzpicture}
\end{figure}
\end{frame}



\begin{frame}[fragile]
\frametitle{Lossless Graph Compression}
\tikzstyle{vertex}=[circle,draw,minimum size=20pt,inner sep=0pt]
\tikzstyle{edge} = [draw,thick,-]

\begin{figure}
\begin{tikzpicture}[scale=1.8, auto,swap]
    % Draw a 7,11 network
    % First we draw the vertices
    \foreach \pos/\name in {{(0,1)/1}, {(2,2)/2}, {(4,2)/4},
      {(2,0)/3}, {(4,0)/5}, {(6,2)/6}, {(6,0)/7},
      {(8,1)/8}}
        \node[vertex] (\name) at \pos {$\name$};
    % Connect vertices with edges and draw weights
    \foreach \source/ \dest in {2/1, 4/2,3/1,3/2,
                                3/4,     5/4,5/3,
                                         6/4,6/5, 6/8,
                                         7/5,7/6, 7/8}
        \path[edge] (\source) --  (\dest);
\end{tikzpicture}
\end{figure}
\end{frame}

\begin{frame}[fragile]
\frametitle{Gap representation}
\begin{block}{Delta}
\begin{enumerate}
\item
      adjacency list of \(v\)
\item
      store difference with previous vertex
\item
      store difference with \(v\)
\end{enumerate}
\end{block}
\pause
\begin{block}{Example adj\((3)\)}
-2, 1, 2, 1
\end{block}
\pause
\begin{block}{Assumptions?}
\pause
Neighbors of a vertex are close to the vertex.
\end{block}

\end{frame}

\begin{frame}[fragile]
\frametitle{Reference compression}
\begin{block}{\(N(x)\) and \(N(y)\)}
\begin{enumerate}
\item
      \(N(y)\) is a previous vertex
\item

      which elements of \(N(x)\) are not in \(N(y)\)?
\item
      \(N(x) \setminus N(y)\)
\end{enumerate}
\end{block}
\pause
\begin{block}{Example adj\((4)\)}
adj\((2)\) = [1, 3, 4];
adj\((4)\) = [2, 3, 5, 6]\(\rightarrow\) \(\langle 2, 101_{2}, [1, 1]\rangle\)\\
using the triple \(\langle\) previous vertex, characteristic vector of the
vertices of \(N(y)\) that are not in \(N(x)\), the encoding of \(N(x) \setminus N(y)\rangle\)
\end{block}
\pause
\begin{block}{Assumptions?}
\pause
\(N(x)\) and \(N(y)\) are almost identical
\end{block}
\end{frame}


\begin{frame}[fragile]
\frametitle{Interval encoding}
\pause
\begin{block}{\([b,e]\)}
\begin{itemize}
      \item
      \([b, e - b]\)
      \item if all intervals are longer than a threshold \(L\) \(\Rightarrow\)
      decrement all lengths by \(L\)
\end{itemize}
\end{block}
\end{frame}

\begin{frame}[fragile]
\frametitle{Figures}

        \begin{itemize}
          \item
David Eppstein, Public Domain, \url{https://commons.wikimedia.org/w/index.php?curid=10261635}
        \end{itemize}
\end{frame}

\begin{frame}[containsverbatim]\frametitle{License}
  \small
This work is licensed under a Creative Commons Attribution-ShareAlike
4.0 International License
(\url{https://creativecommons.org/licenses/by-sa/4.0/}).

\faCreativeCommons\ \faCreativeCommonsBy\ \faCreativeCommonsSa
\end{frame}
\end{document}
